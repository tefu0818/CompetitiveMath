\documentclass[a4paper,12pt]{jsarticle}
\usepackage{amsmath,amssymb,layout,enumerate,bm,verbatim}
% \usepackage{)  ※パッケージ追加用
\usepackage{ascmac}

%%%%%%%%%
\usepackage[dvips,dvipdfm]{graphicx}

\setlength{\oddsidemargin}{-15truemm}
\setlength{\topmargin}{-20truemm}
\setlength{\textwidth}{49zw}
\setlength{\textheight}{40\baselineskip}
\addtolength{\textheight}{\topskip}

\setlength{\headheight}{0zw}
\setlength{\headsep}{0zw}

% \setlength{\columnseprule}{0.4pt} % 中央の線の太さの設定
\setlength{\parindent}{0pt}       % 本文の字下げの設定
% \setlength{\mathindent}{4zw}      % 数式の字下げの設定
\setlength{\fboxrule}{0.5pt}      % boxの枠の太さ

\newcommand{\N}{\mathbb{N}}
\newcommand{\Z}{\mathbb{Z}}
\newcommand{\Q}{\mathbb{Q}}
\newcommand{\R}{\mathbb{R}}
\newcommand{\C}{\mathbb{C}}

\newcommand{\ds}{\displaystyle}

\renewcommand{\labelenumi}{(\arabic{enumi})}
% \renewcommand{\labelenumi}{(\roman{enumi})}
% \renewcommand{\labelenumi}{(\theenumi)}
\renewcommand{\theenumii}{\roman{enumii}}

\newcommand{\p}[2]{{}_{#1}\textrm{P}_{#2}}
\renewcommand{\c}[2]{{}_{#1}\textrm{C}_{#2}}
\newcommand{\h}[2]{{}_{#1}\textrm{H}_{#2}}

\pagestyle{empty}

\begin{document}

\section*{事前演習問題A} %%%%%%%%%%%%%%%%% JJMO問題

\fbox{A1} (基礎)

\begin{enumerate}
\item 
10101010101を1000以上の2つの自然数の積で表せ。

\item 
実数$x,y$が$(x+\sqrt{1+x^2})(y+\sqrt{1+y^2})=1$をみたすとき$x+y$の値を求めよ。 

\end{enumerate}

\bigskip

\fbox{A2} (標準)

\begin{enumerate}
\item 
  \begin{enumerate}
  \item $(pr+qs)^{2} + (ps-qr)^{2} = (p^{2}+q^{2})(r^{2}+s^{2})$\ を示せ。
  \item \vspace{0.5zw}
  {\small $\ds \frac{(2^{2} + 2 - 1 )^{2} + ( 2 \cdot 2 + 1 )^{2}}{(1^{2} + 1 - 1 )^{2} + ( 2 \cdot 1 + 1 )^{2}} \cdot \frac{(4^{2} + 4 - 1 )^{2} + ( 2 \cdot 4 + 1 )^{2}}{(3^{2} + 3 - 1 )^{2} + ( 2 \cdot 3 + 1 )^{2}} \cdot \frac{(6^{2} + 6 - 1 )^{2} + ( 2 \cdot 6 + 1 )^{2}}{(5^{2} + 5 - 1 )^{2} + ( 2 \cdot 5 + 1 )^{2}} \cdot \  \\ \hfill \cdots \ \cdot \frac{(14^{2} + 14 - 1 )^{2} + ( 2 \cdot 14 + 1 )^{2}}{(13^{2} + 13 - 1 )^{2} + ( 2 \cdot 13 + 1 )^{2}}\cdot \frac{(16^{2} + 16 - 1 )^{2} + ( 2 \cdot 16 + 1 )^{2}}{(15^{2} + 15 - 1 )^{2} + ( 2 \cdot 15 + 1 )^{2}}$}\\
の値を求めよ。
  \end{enumerate}
  
\item 
以下の連立方程式を満たす実数$x,y,z$の組をすべて求めよ。
\begin{equation}
  \left\{
    \begin{alignedat}{4}
      x^{2} + 9  =&\ &  & \  &   2y & \ &+ &\,2z   \\ 
      y^{2} + 4  =&  &x &    &+\,2y &   &+ &\,z   \\
      z^{2} + 1  =&  &x &    &      &   &+ &\,3z  \nonumber
    \end{alignedat}
  \right.
\end{equation}

\end{enumerate}

\fbox{A3} (応用)

\begin{enumerate}
\item 
  \begin{enumerate}
  \item $a(x-2)^{2}+b(x-2)+c$ を展開せよ。
  \item 実数の組$(p,q,r)$について,$(p,q,r)を(p,-4p+q,4p-2q+r)$に変える操作を$A$,$(p,q,r)を(p+q+r,2p+q,p)$に変える操作を$B$とする。例えば$(1,2,3)$に操作$B$を行うと$(6,4,1)$となり,さらに操作$A$を行うと$(6,-20,17)$となる。$(1,1,1)$に操作$A,B$を何度か行うことで$(0,0,2017)$に変えることができるか。できるならばその方法を示し,できないならばそれを証明せよ。
  \end{enumerate}
\item 
  \begin{enumerate}
  \item 実数$x,y,z$に対して,
    $$ (x+y+z)^{2} \geq (2x+z)(2y+z) $$
  が成り立つことを示せ。
  \item 正の実数$a,b,c$に対して,
    $$ (2a+b+c)(a+2b+c)(a+b+2c) \geq (3a+b)(3b+c)(3c+a) $$
  が成り立つことを示せ。
  \end{enumerate}
\end{enumerate}

\newpage

\section*{事前演習問題A} %%%%%%%%%%%%%%%%% JMO問題

\fbox{A1} (基礎)

\begin{enumerate}
\item 
以下の連立方程式を満たす実数$x,y,z$の組をすべて求めよ。
\begin{eqnarray}
  \left\{
    \begin{array}{l}
      x^{4} + 6y^{2} + 1 = 4(z^{3} + z)\vspace{0.5zw} \\ 
      y^{4} + 6z^{2} + 1 = 4(x^{3} + x) \vspace{0.5zw} \\
      z^{4} + 6x^{2} + 1 = 4(y^{3} + y) \nonumber
    \end{array}
  \right.
\end{eqnarray}

\item \vspace{0.5zw}
実数$x,y$が$(x+\sqrt{1+x^2})(y+\sqrt{1+y^2})=1$をみたすとき$x+y$の値を求めよ。
\end{enumerate}

\bigskip

\fbox{A2} (標準)

\begin{enumerate}
\item 
以下の連立方程式を解け。
\begin{eqnarray}
  \left\{
    \begin{array}{l}
      \dfrac{y}{x+y} = x+2018 \vspace{0.5zw} \\ 
      \dfrac{y}{2x+y+1} = 2x+2019 \vspace{0.5zw} \\
      \dfrac{y}{3x+y+2} = 3x+2020 \nonumber
    \end{array}
  \right.
\end{eqnarray}

\item \vspace{0.5zw}
  {\small $\ds \frac{(2^{2} + 2 - 1 )^{2} + ( 2 \cdot 2 + 1 )^{2}}{(1^{2} + 1 - 1 )^{2} + ( 2 \cdot 1 + 1 )^{2}} \cdot \frac{(4^{2} + 4 - 1 )^{2} + ( 2 \cdot 4 + 1 )^{2}}{(3^{2} + 3 - 1 )^{2} + ( 2 \cdot 3 + 1 )^{2}} \cdot \frac{(6^{2} + 6 - 1 )^{2} + ( 2 \cdot 6 + 1 )^{2}}{(5^{2} + 5 - 1 )^{2} + ( 2 \cdot 5 + 1 )^{2}} \cdot \  \\ \hfill \cdots \ \cdot \frac{(14^{2} + 14 - 1 )^{2} + ( 2 \cdot 14 + 1 )^{2}}{(13^{2} + 13 - 1 )^{2} + ( 2 \cdot 13 + 1 )^{2}}\cdot \frac{(16^{2} + 16 - 1 )^{2} + ( 2 \cdot 16 + 1 )^{2}}{(15^{2} + 15 - 1 )^{2} + ( 2 \cdot 15 + 1 )^{2}}$}\\
の値を求めよ。

\item 
実数に対して定義され実数値をとる狭義単調増加関数$(p<q\Rightarrow f(p)<f(q))$であって,任意の実数$x$に対して,
  $$ f(f(f(x))) = x $$
が成り立つものをすべて求めよ。

\item 
実数$x,y$が$(x+\sqrt{1+y^2})(y+\sqrt{1+x^2})=1$をみたすとき$x+y$の値を求めよ。 
\end{enumerate}

\fbox{A3} (応用)

\begin{enumerate}
\item 
  \begin{enumerate}
  \item
  正の実数$a,b,c$について,
    $$ \frac{a^{2}}{b} + \frac{b^{2}}{c} + \frac{c^{2}}{a} \geq a + b + c $$
  が成り立つことを示せ。
  \item 
  正の実数$a,b,c$について,
    $$ \frac{a^{4}}{b^{3}} + \frac{b^{4}}{c^{3}} + \frac{c^{4}}{a^{3}} \geq \frac{a^{2}}{b} + \frac{b^{2}}{c} + \frac{c^{2}}{a} $$
  が成り立つことを示せ。
  \end{enumerate}
\item 
実数に対して定義され実数値をとる関数$f$であって,任意の実数$x,y$に対して,
  $$ f(x+f(y)) = f(f(x)-y) + 2y $$
が成り立つものをすべて求めよ。
\end{enumerate}

\newpage

\section*{JJMO事前演習問題A ---解答---} %%%%%%%%%%%%%%%%% JJMO解答

\fbox{A1}

\begin{enumerate}
\item 
10101010101を$x$とすると,
  $$ 99x = 999999999999 = 10^{12}-1 = (10^{6}-1)(10^{6}+1) = 999999 \times 1000001$$
よって,
  $$ x= \frac{999999}{99} \times 1000001 = 10101 \times 1000001$$
が答の一例。
\item 
$(x-\sqrt{1+x^{2}})(x+\sqrt{1+x^2}) = -1$に注意して,与式の両辺に\ $x-\sqrt{1+x^{2}}$\ をかけると,
$$ -(y+\sqrt{1+y^{2}}) = x-\sqrt{1+x^{2}} \ \Leftrightarrow \ x+y = \sqrt{1+x^{2}} - \sqrt{1+y^{2}} $$
同様に,与式の両辺に\ $y-\sqrt{1+y^{2}}$\ をかけると,
$$ -(x+\sqrt{1+x^{2}}) = y-\sqrt{1+y^{2}} \ \Leftrightarrow \ x+y = -\sqrt{1+x^{2}} + \sqrt{1+y^{2}} $$
ゆえに,
$$ x+y = 0 $$
\end{enumerate}

\bigskip

\fbox{A2}

\begin{enumerate}
\item 
  \begin{enumerate}
  \item 
  左辺右辺ともに展開すれば\ $p^{2}r^{2} + q^{2}s^{2} + p^{2}s^{2} + q^{2}r^{2}$\ になる。
  \item (i)において,$p = n+1 ,q = -1 ,r = n ,s = 1$を代入して,
  $$ (n^{2} + n -1)^{2} + (2n + 1)^{2} = (n^{2} + 1) \left\{( n + 1 )^{2} + 1 \right\} $$
  これを用いれば,$f(n) = n^{2} + 1$として,
  \begin{eqnarray} (与式) & = & \frac{f(2)f(3)}{f(1)f(2)} \cdot \frac{f(4)f(5)}{f(3)f(4)} \cdot \frac{f(6)f(7)}{f(5)f(6)} \cdot \ \cdots \ \cdot \frac{f(14)f(15)}{f(13)f(14)} \cdot \frac{f(16)f(17)}{f(15)f(16)} \nonumber \\
  & = & \frac{f(17)}{f(1)} = \frac{17^{2}+1}{1^{2}+1} = 145 \nonumber
  \end{eqnarray}

  \end{enumerate}

\item
与えられた三式をすべて足すと,
  $$ x^{2}+y^{2}+z^{2}+1+4+9 = 2x + 4y + 6z \ \Leftrightarrow \ (x-1)^{2} + (y-2)^{2} + (z-3)^{2} = 0 $$
実数を二乗して得られる値は必ず0以上なので,これが成り立つためには,
  $$ x-1 = y-2 = z-3 =0 \ \Leftrightarrow\ (x,y,z) = (1,2,3)$$
が必要。逆に$(x,y,z) = (1,2,3)$が元の連立方程式を満たすことはすぐに確かめられる。


\end{enumerate}

\fbox{A3}
\begin{enumerate}
\item 
  \begin{enumerate}
  \item
    $$ a(x-2)^{2}+b(x-2)+c = ax^{2} + (-4a+b)x + 4a-2b+c$$
  \item
  それぞれの操作で$q^{2}-4pr$が不変であることに注意する。実際,
    $$(-4p+q)^{2} - 4p(4p-2q+r) = q^{2}-4pr$$
    $$(2p+q)^{2} - 4(p+q+r)p = q^{2}-4pr$$
  問題文の具体例$(1,2,3)\xrightarrow[]{B} (6,4,1) \xrightarrow[]{A} (6,-20,17)$の場合だと,確かに
    $$ 2^{2}-4\cdot 1\cdot 3 = 4^{2}-4\cdot 6\cdot 1 = (-20)^{2}-4\cdot 6\cdot 17 \ \ (= -8)$$
  となっている。ここで,
    $$1^{2}-4\cdot 1\cdot 1 \neq 0^{2}-4\cdot 0\cdot 2017$$
  であるので,$(1,1,1)\rightarrow \cdots \rightarrow (0,0,2017)$とすることはできない。
  \end{enumerate}
\item
  \begin{enumerate}
  \item 
    $$ (x+y+z)^{2} - (2x+z)(2y+z) = x^{2} + 2xy + y^{2} - 4xy = (x-y)^{2} \geq 0$$
  よりよい。
  \item
  (i)において,$(x,y,z) = (a,b,b+c)$とすれば,
    $$ (a+2b+c)^{2} \geq (2a+b+c)(3b+c) $$
  同様にして,$(x,y,z) = (b,c,c+a),(c,a,a+b)$とすれば,
    $$ (a+b+2c)^{2} \geq (a+2b+c)(3c+a) $$
    $$ (2a+b+c)^{2} \geq (a+b+2c)(3a+b) $$
  これら三式を辺々かけて,
    $$ (2a+b+c)^{2}(a+2b+c)^{2}(a+b+2c)^{2} \geq (3a+b)(3b+c)(3c+a)(2a+b+c)(a+2b+c)(a+b+2c)$$
  両辺$(2a+b+c)(a+2b+c)(a+b+2c)>0$で割れば,
    $$ (2a+b+c)(a+2b+c)(a+b+2c) \geq (3a+b)(3b+c)(3c+a) $$
  を得る。
  \end{enumerate}
 
\end{enumerate}

\newpage

\section*{JMO事前演習問題A ---解答---} %%%%%%%%%%%%%%%%% JMO解答

\fbox{A1}
\begin{enumerate}

\item 
3式を辺々足して整理すると,
  $$ (x-1)^{4} + (y-1)^{4} + (z-1)^{4} = 0 $$
$x,y,z$が実数のとき左辺の各項は0以上であることに注意して,$(x,y,z)=(1,1,1)$に限られる。逆に$(x,y,z)=(1,1,1)$は与方程式を満たすのでこれが求めるものである。

\item 
$(x-\sqrt{1+x^{2}})(x+\sqrt{1+x^2}) = -1$に注意して,与式の両辺に\ $x-\sqrt{1+x^{2}}$\ をかけると,
$$ -(y+\sqrt{1+y^{2}}) = x-\sqrt{1+x^{2}} \ \Leftrightarrow \ x+y = \sqrt{1+x^{2}} - \sqrt{1+y^{2}} $$
同様に,与式の両辺に\ $y-\sqrt{1+y^{2}}$\ をかけると,
$$ -(x+\sqrt{1+x^{2}}) = y-\sqrt{1+y^{2}} \ \Leftrightarrow \ x+y = -\sqrt{1+x^{2}} + \sqrt{1+y^{2}} $$
ゆえに,
$$ x+y = 0 $$
\end{enumerate}

\bigskip


\fbox{A2}

\begin{enumerate}
\item 
与式を満たす$(x,y)$について,$\alpha = x , \beta = 2x+1 , \gamma = 3x+2$とすれば,
\begin{eqnarray}
  \left\{
    \begin{array}{l}
      \dfrac{y}{\alpha+y} = \alpha+2018 \vspace{0.5zw} \\ 
      \dfrac{y}{\beta+y} = \beta+2018 \vspace{0.5zw} \\
      \dfrac{y}{\gamma+y} = \gamma+2018  \\\nonumber
    \end{array}
  \right.
  \ \Rightarrow \  \left\{
    \begin{array}{l}
      \alpha^{2} + (y+2018)\alpha + 2017y = 0 \vspace{0.5zw} \\ 
      \beta^{2} + (y+2018)\beta + 2017y = 0 \vspace{0.5zw} \\
      \gamma^{2} + (y+2018)\gamma + 2017y = 0  \nonumber
    \end{array}
  \right.
\end{eqnarray}
すなわち,$\alpha,\beta,\gamma$は$t$についての二次方程式$t^{2}+(y+2018)t+2017y=0$の解である。ここで二次方程式の解は高々2つしか存在しないので$\alpha,\beta,\gamma$のうち少なくとも2つは等しいことがわかり,$\alpha=\beta,\beta=\gamma,\gamma=\alpha$,いずれの場合も$x=-1$を得る。$x=-1$を与式に代入して,
$$ \frac{y}{y-1} = 2017 \Leftrightarrow y = \frac{2017}{2016} $$ 
ゆえに,$(x,y) = \left(-1,\,\dfrac{2017}{2016}\right)$であり,これはたしかに与式を満たす。

\item 
  $$ (n^{2} + n -1)^{2} + (2n + 1)^{2} = (n^{2} + 1) \left\{( n + 1 )^{2} + 1 \right\} $$
であるので,$f(n) = n^{2} + 1$として,
\begin{eqnarray} (与式) & = & \frac{f(2)f(3)}{f(1)f(2)} \cdot \frac{f(4)f(5)}{f(3)f(4)} \cdot \frac{f(6)f(7)}{f(5)f(6)} \cdot \ \cdots \ \cdot \frac{f(14)f(15)}{f(13)f(14)} \cdot \frac{f(16)f(17)}{f(15)f(16)} \nonumber \\
& = & \frac{f(17)}{f(1)} = \frac{17^{2}+1}{1^{2}+1} = 145 \nonumber
\end{eqnarray}

\item
ある$y$について,$y<f(y)$となったと仮定すると,$f$の単調増加性$(p<q\Rightarrow f(p)<f(q))$から,
  $$ f(y)<f(f(y)) $$
  $$ f(f(y))< f(f(f(y))) $$
このとき,
  $$ y<f(y)<f(f(y))<f(f(f(y)))=y$$
となり,矛盾。同様に$y>f(y)$となったと仮定すると,$y>f(y)>f(f(y))>f(f(f(y)))=y$となり,矛盾。ゆえに任意の$x$について,$f(x)=x$。逆に$f(x)=x$が題意を満たすことはすぐに分かる。

\item
実数$x,y$について,$\ds x = \frac{s-s^{-1}}{2},y = \frac{t-t^{-1}}{2}なる正の実数s,t$がただ一つ存在する。(グラフを書けばわかる。)これらの$s,t>0$に対して,
  $$ \sqrt{1+\left(\frac{s-s^{-1}}{2}\right)^{2}} = \frac{s+s^{-1}}{2},\ \ \sqrt{1+\left(\frac{t-t^{-1}}{2}\right)^{2}} = \frac{t+t^{-1}}{2}$$
に注意すれば,
\begin{eqnarray*}
  (x+\sqrt{1+y^2})(y+\sqrt{1+x^2}) = 1 &\Leftrightarrow& \left( \frac{s-s^{-1}}{2} + \frac{t+t^{-1}}{2} \right) \left( \frac{t-t^{-1}}{2} + \frac{s+s^{-1}}{2} \right) = 1 \\
  &\Leftrightarrow& \left( s+t-\frac{1}{s}+\frac{1}{t} \right) \left( s+t+\frac{1}{s}-\frac{1}{t} \right) = 4 \\
  &\Leftrightarrow& s^{2}t^{2}(s+t)^{2} - (t-s)^{2} = 4s^{2}t^{2} \\
  &\Leftrightarrow& (st-1) \bigl\{ (s+t)^{2}st + (s-t)^{2} \bigr\} = 0 \\
  &\Leftrightarrow& st = 1 \qquad (\ \because (s+t)^{2}st + (s-t)^{2}>0)
\end{eqnarray*}
ゆえに,
  $$ x+y = \frac{s-s^{-1}}{2} + \frac{t-t^{-1}}{2} = \frac{s-t}{2} + \frac{t-s}{2} = 0$$
\end{enumerate}

\fbox{A3}

\begin{enumerate}
\item 

  \begin{enumerate}
  \item
  コーシー・シュワルツの不等式より,
    $$ \left(\frac{a^{2}}{b} + \frac{b^{2}}{c} + \frac{c^{2}}{a}\right)(b+c+a) \geq \left(\sqrt{\frac{a^{2}}{b}\cdot b} + \sqrt{\frac{b^{2}}{c}\cdot c} + \sqrt{\frac{c^{2}}{a}\cdot a}\right)^{2} = (a+b+c)^{2} $$
  両辺を$a+b+c>0$で割って,
    $$ \frac{a^{2}}{b} + \frac{b^{2}}{c} + \frac{c^{2}}{a} \geq a + b + c $$
  を得る。\\

  \hspace{-1.5zw} (別解1)\\
  Engel型のコーシー・シュワルツの不等式より,
    $$ \frac{a^{2}}{b} + \frac{b^{2}}{c} + \frac{c^{2}}{a} \geq \frac{(a+b+c)^{2}}{b+c+a} = a + b + c $$ \\

  \hspace{-1.5zw} (別解2)\\
  相加相乗平均の不等式より,
    $$ \frac{\dfrac{a^{2}}{b}+\dfrac{a^{2}}{b}+\dfrac{a^{2}}{b}+\dfrac{a^{2}}{b}+\dfrac{b^{2}}{c}+\dfrac{b^{2}}{c}+\dfrac{c^{2}}{a}}{7} \geq \sqrt[7]{\dfrac{a^{2}}{b}\dfrac{a^{2}}{b}\dfrac{a^{2}}{b}\dfrac{a^{2}}{b}\dfrac{b^{2}}{c}\dfrac{b^{2}}{c}\dfrac{c^{2}}{a}} = a $$
  同様にして,
  \begin{eqnarray}
  \frac{\dfrac{b^{2}}{c}+\dfrac{b^{2}}{c}+\dfrac{b^{2}}{c}+\dfrac{b^{2}}{c}+\dfrac{c^{2}}{a}+\dfrac{c^{2}}{a}+\dfrac{a^{2}}{b}}{7} \geq b \nonumber \\
\frac{\dfrac{c^{2}}{a}+\dfrac{c^{2}}{a}+\dfrac{c^{2}}{a}+\dfrac{c^{2}}{a}+\dfrac{a^{2}}{b}+\dfrac{a^{2}}{b}+\dfrac{b^{2}}{c}}{7} \geq c \nonumber
  \end{eqnarray}
  上3式を足せば,
    $$ \frac{a^{2}}{b} + \frac{b^{2}}{c} + \frac{c^{2}}{a} \geq a + b + c $$
を得る。\\

\hspace{-1.5zw} (別解3)\\
相加相乗平均の不等式より,
  $$ \frac{a^{2}}{b} + b \geq 2\sqrt{\frac{a^{2}}{b}\cdot b} = 2a \hspace{3zw} \therefore \ \frac{a^{2}}{b} \geq  2a - b $$ 
同様にして,
\begin{eqnarray}
\frac{b^{2}}{c} \geq  2b - c \nonumber\\
\frac{c^{2}}{a} \geq  2c - a \nonumber
\end{eqnarray}
上3式を足せば,
  $$ \frac{a^{2}}{b} + \frac{b^{2}}{c} + \frac{c^{2}}{a} \geq a + b + c $$
  を得る。\\

  \item 
  Engel型のコーシー・シュワルツの不等式より,
    $$ \frac{a^{4}}{b^{3}} + \frac{b^{4}}{c^{3}} + \frac{c^{4}}{a^{3}} = \frac{\left(\dfrac{a^{2}}{b}\right)^{2}}{b} + \frac{\left(\dfrac{b^{2}}{c}\right)^{2}}{c} + \frac{\left(\dfrac{c^{2}}{a}\right)^{2}}{a} \geq \frac{\left(\dfrac{a^{2}}{b} + \dfrac{b^{2}}{c} + \dfrac{c^{2}}{a}\right)^{2}}{b+c+a} $$
  さらにこの右辺について,(1)より,
    $$ \frac{\left(\dfrac{a^{2}}{b} + \dfrac{b^{2}}{c} + \dfrac{c^{2}}{a}\right)^{2}}{b+c+a} \geq \frac{\left(\dfrac{a^{2}}{b} + \dfrac{b^{2}}{c} + \dfrac{c^{2}}{a}\right)(a+b+c)}{b+c+a} = \dfrac{a^{2}}{b} + \dfrac{b^{2}}{c} + \dfrac{c^{2}}{a} $$
  となるのでよい。
  \end{enumerate}

\item 
与式において,$x$に$-f(y)$を代入して,
$$ f(-f(y)+f(y)) = f\bigl(f(-f(y))-y\bigr) + 2y \ \Leftrightarrow \ f\bigl(f(-f(y))-y\bigr) = 2y - f(0)$$
$y$が実数全体を動くとき,右辺の$2y-f(0)$も実数全体を動くので$f$は全射(任意の実数$s$について,$f(t)=s$なる実数$t$が存在する)。ゆえに,ある実数$\alpha$が存在して$f(\alpha)=0$。\\
さらに,与式において$y$に$\alpha$を代入して,
  $$ f(x+f(\alpha)) = f(f(x)-\alpha) + 2\alpha \ \Leftrightarrow \ f(x) - \alpha - \alpha= f(f(x)-\alpha)$$
ここで$f$の全射性から$x$が全実数を動くとき$f(x)-\alpha$も全実数を動くので,これを$z$と置き換えて,
$$ 任意の実数zについて,z - \alpha = f(z)$$
がわかる。すなわち,題意を満たす$f$は$c$を定数として,$f(x)=x+c$の形に限られる。逆に$f(x)=x+c$が与式を満たすことはすぐにわかるのでこれが求めるものである。
\end{enumerate}
\vfil






%%%%%%%%%%%%%%%%%%%%%%%%%%%%%%%%%%%%%%%%%%%%%%%%%%%%%%おまけ
(参考)
\begin{itembox}[l]{重み付き相加相乗平均の不等式}
$p_{1},p_{2},\,\cdots ,p_{n} > 0 $が$p_{1}+p_{2}+\dots + p_{n} =1$をみたすとき,任意の$a_{1},a_{2},\cdots ,a_{n} > 0 $について,
\begin{equation}
p_{1}a_{1} + p_{2}a_{2} + \cdots + p_{n}a_{n} \geq a_{1}^{p_{1}}  a_{2}^{p_{2}}  \cdots a_{n}^{p_{n}} \nonumber
\end{equation}
\end{itembox}\\

\begin{itembox}[l]{コーシー・シュワルツの不等式}
実数$a_{1},a_{2},\cdots ,a_{n} , \,b_{1},b_{2},\cdots ,b_{n}$ について,
\begin{equation}
 (a_{1}^{2} + a_{2}^{2} + \cdots + a_{n}^{2})(b_{1}^{2} + b_{2}^{2} + \cdots + b_{n}^{2}) \geq (a_{1}b_{1} + a_{2}b_{2} + \cdots + a_{n}b_{n})^{2}  \nonumber
\end{equation}
\end{itembox}\\

\begin{itembox}[l]{有用な不等式(Engel型コーシー・シュワルツ)}
正の実数$a_{1},a_{2},\cdots ,a_{n} , \,x_{1},x_{2},\cdots ,x_{n}$ について,
\begin{equation}
 \frac{x_{1}^{2}}{a_{1}} + \frac{x_{2}^{2}}{a_{2}} + \cdots +\frac{x_{n}^{2}}{a_{n}} \geq \frac{(x_{1}+x_{2}+ \cdots +x_{n})^{2}}{a_{1}+a_{2}+ \cdots +a_{n}} \nonumber
\end{equation}
\end{itembox}

※ コーシー・シュワルツの不等式と(ほとんど)同値。実際,コーシー・シュワルツの不等式において,

$\ds a_{i} \rightarrow \sqrt{\frac{x_{i}^{2}}{{a_{i}}}} \ , \  b_{i} \rightarrow \sqrt{a_{i}}\nonumber
$
とすれば,Engel型が得られる。\\

(その他) 

凸不等式,ヘルダー(H\"older)の不等式,シューア(Schur)の不等式,並び替え不等式,チェビシェフ(Chebyshev)の不等式,べき平均不等式,ミンコフスキー(Minkowski)の不等式
\end{document}


