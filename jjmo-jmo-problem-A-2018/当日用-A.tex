\documentclass[a4paper,12pt]{jsarticle}
\usepackage{amsmath,amssymb,layout,enumerate,bm,verbatim}
% \usepackage{)  ※パッケージ追加用
\usepackage{ascmac}

%%%%%%%%%
\usepackage[dvips,dvipdfm]{graphicx}

\setlength{\oddsidemargin}{-15truemm}
\setlength{\topmargin}{-20truemm}
\setlength{\textwidth}{49zw}
\setlength{\textheight}{40\baselineskip}
\addtolength{\textheight}{\topskip}

\setlength{\headheight}{0zw}
\setlength{\headsep}{0zw}

% \setlength{\columnseprule}{0.4pt} % 中央の線の太さの設定
\setlength{\parindent}{0pt}       % 本文の字下げの設定
% \setlength{\mathindent}{4zw}      % 数式の字下げの設定
\setlength{\fboxrule}{0.5pt}      % boxの枠の太さ

\newcommand{\N}{\mathbb{N}}
\newcommand{\Z}{\mathbb{Z}}
\newcommand{\Q}{\mathbb{Q}}
\newcommand{\R}{\mathbb{R}}
\newcommand{\C}{\mathbb{C}}

\newcommand{\ds}{\displaystyle}

\renewcommand{\labelenumi}{(\arabic{enumi})}
% \renewcommand{\labelenumi}{(\roman{enumi})}
% \renewcommand{\labelenumi}{(\theenumi)}
\renewcommand{\theenumii}{\roman{enumii}}

\newcommand{\p}[2]{{}_{#1}\textrm{P}_{#2}}
\renewcommand{\c}[2]{{}_{#1}\textrm{C}_{#2}}
\newcommand{\h}[2]{{}_{#1}\textrm{H}_{#2}}

\pagestyle{empty}

\begin{document}

\section*{JJMO演習問題A} %%%%%%%%%%%%%%%%% JJMO問題

\fbox{A1} 
すべてが等しいわけではない3つの実数$x,y,z$が
$$ x+\frac{1}{yz} = y+\frac{1}{zx} = z+\frac{1}{xy} $$
をみたすとき,$x+\dfrac{1}{yz}$の値を求めよ。

\bigskip

\fbox{A2}
\begin{eqnarray*}
  \begin{cases}
    x^{2} + y^{2} = z^{2} \\
    (x+3)^{2} + (y+4)^{2} = (z+5)^{2} 
  \end{cases}
\end{eqnarray*}
を満たす正の実数$x,y,z$について,$\dfrac{x^{2}}{yz}$の値を求めよ。
\bigskip

\fbox{A3}
正の実数$a,b,c$に対し,
$$ (a^{2}+1)(b^{2}+1)(c^{2}+1) \geq (a+b)(b+c)(c+a) $$
が成立することを示せ。

\newpage

\section*{JMO演習問題A} %%%%%%%%%%%%%%%%% JMO問題

\bigskip

\fbox{A1} 
正の実数$a,b,cに対して,$
$$ \frac {a^{2}+1}{b+c}+\frac {b^{2}+1}{c+a}+\frac {c^{2}+1}{a+b} \ge 3$$
が成り立つことを示せ。

\bigskip


\fbox{A2} 
正の実数$x,y,z$に対して,
$$ \frac{x}{(2x+y+z)^{2}}+\frac{y}{(2x+2y+z)^{2}}+\frac{z}{(2x+2y+2z)^{2}} < \frac{1}{2x+2y+2z}$$
が成り立つことを示せ。

\bigskip


\fbox{A3}
実数に対して定義され実数値をとる関数であって,任意の実数$x,y$について,
$$ f(x^{4}+f(y)) = f(x)^{4}+y$$
をみたすものをすべて求めよ。
\newpage

\section*{JJMO演習問題A ---解答---} %%%%%%%%%%%%%%%%% JJMO解答

\fbox{A1}

$x\neq0, y \neq 0, z \neq 0$のもとで,
$$ x+\frac{1}{yz} = y+\frac{1}{zx} \Leftrightarrow x-y = \frac{y-x}{xyz} \Leftrightarrow (xyz+1)(x-y) = 0 $$
同様に,
$$ (xyz+1)(y-z) = 0,\ (xyz+1)(z-x) = 0 $$
ここで,$xyz \neq -1$とすると,$x=y=z$となり矛盾。よって,$xyz = -1$。ゆえに,
$$ x+\frac{1}{yz} = x+(-x) = 0$$

\bigskip

\fbox{A2}

第二式を展開すると,第一式より,
\begin{eqnarray*}
  (x+3)^{2} + (y+4)^{2} = (z+5)^{2} & \Leftrightarrow & x^{2}+6x+9 + y^{2}+8y+16 = z^{2}+10z+25 \\
  & \Leftrightarrow & 3x+4y = 5z
\end{eqnarray*}
再び第一式を用いて,
$$(3x+4y)^{2} = 25z^{2} = 25(x^{2}+y^{2})$$
ここで,
\begin{eqnarray*}
  (3x+4y)^{2} = 25(x^{2}+y^{2}) & \Leftrightarrow & 16x^{2}-24xy+9y^{2} = 0 \\
  & \Leftrightarrow & (4x-3y)^{2} = 0
\end{eqnarray*}
より,$y=\dfrac{4}{3}x$。これを第一式に代入して,$z=\dfrac{5}{3}x$もわかる。ゆえに,
$$ \frac{x^{2}}{yz} = \frac{x^{2}}{\ \dfrac{4}{3}x\cdot\dfrac{5}{3}x\ } = \frac{9}{20}$$

($\rightarrow$図形的に考えてもできます。)
\bigskip

\fbox{A3}
$$(a^{2}+1)(b^{2}+1)-(a+b)^{2} = a^{2}b^{2}-2ab+1 = (ab-1)^{2} \geq 0$$
などから,
\begin{eqnarray*}
  (a^{2}+1)(b^{2}+1) &\geq& (a+b)^{2} \\
  (b^{2}+1)(c^{2}+1) &\geq& (b+c)^{2} \\
  (c^{2}+1)(a^{2}+1) &\geq& (c+a)^{2} 
\end{eqnarray*}
この三式を辺々かけあわせて,($a,b,c>0$に注意して)ルートをとれば,
$$ (a^{2}+1)(b^{2}+1)(c^{2}+1) \geq (a+b)(b+c)(c+a) $$
を得る。

\newpage

\section*{JMO演習問題A ---解答---} %%%%%%%%%%%%%%%%% JMO解答

\fbox{A1}

相加相乗平均の不等式より,$a^{2}+1 \geq 2\sqrt{a^{2}} = 2a$ などが成り立つので,
$$\frac {a^{2}+1}{b+c}+\frac {b^{2}+1}{c+a}+\frac {c^{2}+1}{a+b} \geq \frac {2a}{b+c}+\frac {2b}{c+a}+\frac {2c}{a+b}$$
さらにこの右辺について,Engel型のコーシー・シュワルツの不等式より,
\begin{eqnarray*}
  \frac {2a}{b+c}+\frac {2b}{c+a}+\frac {2c}{a+b} & = & \frac {a^{2}}{\frac{ab+ac}{2}}+\frac {b^{2}}{\frac{bc+ba}{2}}+\frac {c^{2}}{\frac{ca+cb}{2}} \\
  & \geq & \frac{(a+b+c)^{2}}{\frac{ab+ac}{2}+\frac{bc+ba}{2}+\frac{ca+cb}{2}} \\
  & = & \frac{(a+b+c)^{2}}{ab+bc+ca}
\end{eqnarray*}
よって,
$$\frac{(a+b+c)^{2}}{ab+bc+ca} \geq 3$$
を示せば十分であるが,これは
$$(a+b+c)^{2}-3(ab+bc+ca) = \frac{1}{2}\{(a-b)^{2}+(b-c)^{2}+(c-a)^{2}\} \geq 0$$
から直ちに示される。
\bigskip


\fbox{A2}

\begin{eqnarray*}
  &&\frac{x}{(2x+y+z)^{2}}+\frac{y}{(2x+2y+z)^{2}}+\frac{z}{(2x+2y+2z)^{2}} \\
  &<& \frac{x}{(x+y+z)(2x+y+z)}+\frac{y}{(2x+y+z)(2x+2y+z)}+\frac{z}{(2z+2y+z)(2x+2y+2z)} \\
  &=& \frac{1}{x+y+z}-\frac{1}{2x+y+z}+\frac{1}{2x+y+z}-\frac{1}{2x+2y+z}+\frac{1}{2x+2y+z}-\frac{1}{2x+2y+2z} \\
  &=& \frac{1}{x+y+z}-\frac{1}{2x+2y+2z} = \frac{1}{2x+2y+2z}
\end{eqnarray*}
よりよい。
\bigskip


\fbox{A3}

与式において$x$に0を代入して,
\begin{equation}
  f(f(y))=f(0)^{4}+y
\end{equation}
この右辺は全実数値をとりうるので$f$は全射。ゆえにある$\alpha$が存在して,$f(\alpha)=0$。\\
また,$f(a)=f(b)$になったと仮定すると,(1)式において$y$に$a,b$をそれぞれ代入して,
\begin{eqnarray*}
  f(f(a)) = f(0)^{4}+a \\
  f(f(b)) = f(0)^{4}+b 
\end{eqnarray*}
つまり
$$f(a)=f(b) \Rightarrow a = b \ \ (fは単射)$$

与式において,$(x,y)\rightarrow(\alpha,\alpha)$として,
\begin{equation}
  f(\alpha^{4}+f(\alpha)) = f(\alpha)^{4}+\alpha \Leftrightarrow f(\alpha^{4}) = \alpha
\end{equation}

与式において,$(x,y)\rightarrow(-\alpha,\alpha)$として,
\begin{equation}
  f((-\alpha)^{4}+f(\alpha)) = f(-\alpha)^{4}+\alpha \Leftrightarrow f(\alpha^{4}) = f(-\alpha)^{4}+\alpha
\end{equation}

(2)(3)より,$f(-\alpha)^{4}=0 \Leftrightarrow f(-\alpha) = 0$。$f(\alpha) = 0$でもあるので,$f$の単射性から,$-\alpha = \alpha \Leftrightarrow \alpha = 0$。つまり,$f(0)=0$であるので,(1)より,
\begin{equation}
  f(f(y)) = y
\end{equation}

与式において,$y$を$f(y)$として,(4)より,
$$ f(x^{4}+f(f(y))) = f(x)^{4}+f(y) \Leftrightarrow  f(x^{4}+y) = f(x)^{4}+f(y)$$
ゆえに$f$は狭義単調増加関数である。実際,
$$z>y \Rightarrow f(z) = f((z-y)+y) = f(\sqrt[4]{z-y}\,)^{4} + f(y) >f(y)$$
(4)式と合わせて(予習用問題と同じ議論をして),求める$f$は$f(x)=x$。
\end{document}


